%保存为UTF-8编码格式
%用xelatex编译
 
\documentclass[UTF8,a4paper,12pt]{ctexart}
\usepackage[left=2.50cm, right=2.50cm, top=2.50cm, bottom=2.50cm]{geometry} %页边距
\CTEXsetup[format={\Large\bfseries}]{section} %设置章标题字号为Large,居左
%\CTEXsetup[number={\chinese{section}}]{section}
%\CTEXsetup[name={(,)}]{subsection}
%\CTEXsetup[number={\chinese{subsection}}]{subsection}
%\CTEXsetup[name={(,)}]{subsubsection}
%\CTEXsetup[number=\arabic{subsubsection}]{subsubsection}  %以上四行为各级标题样式设置,可根据需要做修改
 
%\linespread{1.5} %设置全文行间距
 
 
%\usepackage[english]{babel}
%\usepackage{float}     %放弃美学排版图表
\usepackage{fontspec}   %修改字体
\usepackage{amsmath, amsfonts, amssymb} % 数学公式相关宏包
\usepackage{color}      % color content
\usepackage{graphicx}   % 导入图片
\usepackage{subfigure}  % 并排子图
\usepackage{url}        % 超链接
\usepackage{bm}         % 加粗部分公式,比如\bm{aaa}aaa
\usepackage{multirow}
\usepackage{booktabs}
\usepackage{epstopdf}
\usepackage{epsfig}
\usepackage{longtable}  %长表格
\usepackage{supertabular}%跨页表格
\usepackage{algorithm}
\usepackage{algorithmic}
\usepackage{changepage}
\usepackage{listings} % 插入代码用到

% 用来设置附录中代码的样式
\lstset{
	basicstyle          =   \sffamily,          % 基本代码风格
	keywordstyle        =   \bfseries,          % 关键字风格
	commentstyle        =   \rmfamily\itshape,  % 注释的风格,斜体
	stringstyle         =   \ttfamily,  % 字符串风格
	flexiblecolumns,                % 别问为什么,加上这个
	numbers             =   left,   % 行号的位置在左边
	showspaces          =   false,  % 是否显示空格,显示了有点乱,所以不现实了
	numberstyle         =   \zihao{-5}\ttfamily,    % 行号的样式,小五号,tt等宽字体
	showstringspaces    =   false,
	captionpos          =   t,      % 这段代码的名字所呈现的位置,t指的是top上面
	frame               =   lrtb,   % 显示边框
}

\lstdefinestyle{Python}{
	language        =   Python, % 语言选Python
	basicstyle      =   \zihao{-5}\ttfamily,
	numberstyle     =   \zihao{-5}\ttfamily,
	keywordstyle    =   \color{blue},
	keywordstyle    =   [2] \color{teal},
	stringstyle     =   \color{magenta},
	commentstyle    =   \color{red}\ttfamily,
	breaklines      =   true,   % 自动换行,建议不要写太长的行
	columns         =   fixed,  % 如果不加这一句,字间距就不固定,很丑,必须加
	basewidth       =   0.5em,
}

 
 
 
%%%%%%%%%%%%%%%%%%%%%%%
% -- text font --
% compile using Xelatex
%%%%%%%%%%%%%%%%%%%%%%%
% -- 中文字体 --
%\setCJKmainfont{Microsoft YaHei}  % 微软雅黑
%\setCJKmainfont{YouYuan}  % 幼圆
%\setCJKmainfont{NSimSun}  % 新宋体
%\setCJKmainfont{KaiTi}    % 楷体
\setCJKmainfont{SimSun}   % 宋体
%\setCJKmainfont{SimHei}   % 黑体
 
% -- 英文字体 --
\setmainfont{Times New Roman}
%\setmainfont{DejaVu Sans}
%\setmainfont{Latin Modern Mono}
%\setmainfont{Consolas}
%
%
\renewcommand{\algorithmicrequire}{ \textbf{Input:}}     % use Input in the format of Algorithm
\renewcommand{\algorithmicensure}{ \textbf{Initialize:}} % use Initialize in the format of Algorithm
\renewcommand{\algorithmicreturn}{ \textbf{Output:}}     % use Output in the format of Algorithm
\newcommand{\xiaosi}{\fontsize{12pt}{\baselineskip}}     %\xiaosi代替设置12pt字号命令,不加\selectfont,行间距设置无效
\newcommand{\wuhao}{\fontsize{10.5pt}{10.5pt}\selectfont}
 
\usepackage{fancyhdr} %设置全文页眉、页脚的格式
\pagestyle{fancy}
\lhead{}           %页眉左边设为空
\chead{}           %页眉中间
\rhead{}           %页眉右边
%\rhead{\includegraphics[width=1.2cm]{1.eps}}  %页眉右侧放置logo
\lfoot{}          %页脚左边
\cfoot{\thepage}  %页脚中间
\rfoot{}          %页脚右边
 
 
%%%%%%%%%%%%%%%%%%%%%%%
%  设置水印
%%%%%%%%%%%%%%%%%%%%%%%
%\usepackage{draftwatermark}         % 所有页加水印
%\usepackage[firstpage]{draftwatermark} % 只有第一页加水印
% \SetWatermarkText{Water-Mark}           % 设置水印内容
% \SetWatermarkText{\includegraphics{fig/ZJDX-WaterMark.eps}}         % 设置水印logo
% \SetWatermarkLightness{0.9}             % 设置水印透明度 0-1
% \SetWatermarkScale{1}                   % 设置水印大小 0-1
 
\usepackage{hyperref} %bookmarks
\hypersetup{colorlinks, bookmarks, unicode} %unicode
 
 
 
\title{\textbf{\Large{编程作业二报告}}}
\author{涂宇清}
\date{522030910152}
 
 
 
\begin{document}
 
\maketitle
%\tableofcontents
\setcounter{page}{1}        %从下面开始编页,页脚格式为导言部分设置的格式
 
 
\section{对矩形窗信号的采样相关处理}

\subsection{直接采样}
\begin{itemize}
    \item 使用 \verb|X = fft(x, 1000); X = fftshift(X);| 对采样后的离散信号\(x[n]\)作DFT;
    \item 采样所得时域波形\(x[n]\)、其频谱\(X(e^{j\omega})\)的幅度特性\(|X(e^{j\omega})|\)与相位特性\(\angle X(e^{j\omega})\)如下。
\end{itemize}

\begin{figure}[H]   %*表示可跨栏,如果不需要可去掉
  \centering
    \subfigure[直接采样相关图表]{
    \includegraphics[width=17cm]{Q1_1.jpg}}
\end{figure}

\subsection{时移后采样}
\begin{itemize}
  \item 使用 \verb|X1 = fft(x1, 1000); X1 = fftshift(X1);| 对向右平移0.5个单位后采样的离散信号\(x_1[n]\)作DFT;
  \item 采样所得时域波形\(x_1[n]\)、其频谱\(X_1(e^{j\omega})\)的幅度特性\(|X_1(e^{j\omega})|\)与相位特性\(\angle X_1(e^{j\omega})\)如下。
\end{itemize}

\begin{figure}[H]   %*表示可跨栏,如果不需要可去掉
\centering
  \subfigure[时移后采样相关图表]{
  \includegraphics[width=17cm]{Q1_2.jpg}}
\end{figure}

\subsection{时移并滤波后采样}
\begin{itemize}
  \item 使用 \verb|X1(1:450) = 0; X1(551:1000) = 0; x1 = abs(real(ifft(X1)));| 对向右平移0.5个单位后的连续信号频谱作低通滤波(截止频率为\(10\pi Hz\)),并将滤波后信号转化到时域;
  \item 使用 \verb|y = x2(250:50:1000);|对滤波后的连续信号采样,得到离散信号\(y[n]\);
  \item 使用 \verb|Y = fft(y, 1000); Y = fftshift(Y);| 对离散信号\(y[n]\)作DFT;
  \item 采样所得时域波形\(y[n]\)、其频谱\(Y(e^{j\omega})\)的幅度特性\(|Y(e^{j\omega})|\)与相位特性\(\angle Y(e^{j\omega})\)及其滤波前对应特性的对比图如下。
\end{itemize}

\begin{figure}[H]   %*表示可跨栏,如果不需要可去掉
\centering
  \subfigure[时移并滤波后采样相关图表]{
  \includegraphics[width=17cm]{Q1_3.jpg}}
\end{figure}

\subsection{公式推导}
\subsubsection{直接采样}
\begin{align}
	X(e^{j\omega}) &= \sum_{n=-\infty}^{\infty}x[n]e^{-j\omega n} \notag \\
	 				&= \left(\frac{\sin{\frac{9\omega}{2}}}{\sin{\frac{\omega}{2}}}+\cos5\omega\right)e^{-5j\omega} \notag
\end{align}

\subsubsection{时移后采样}
\begin{align}
	X_1(e^{j\omega}) &= \sum_{n=-\infty}^{\infty}x_1[n]e^{-j\omega n} \notag \\
	 				&= \frac{\sin{\frac{11\omega}{2}}}{\sin{\frac{\omega}{2}}}e^{-5j\omega} - 1 \notag
\end{align}

\subsubsection{时移并滤波后采样}
\begin{align}
	Y(e^{j\omega}) &= \sum_{n=-\infty}^{\infty}y[n]e^{-j\omega n} \notag
\end{align}
计算过程同上。
 
\section{对周期性现象的帧采样}
\begin{itemize}
  \item 估算的旋转周期为$75$帧。通过\verb|video.FrameRate|读取原视频帧率,发现原视频帧率为$25$FPS,再通过肉眼观察,发现一个旋转周期约为$3$s,故旋转周期为$3\times 25=75$帧。
\end{itemize}


\end{document}